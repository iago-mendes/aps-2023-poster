\documentclass[landscape, paperwidth=60in, paperheight=40in]{baposter}
% Maximum dimensions: 90in x 48in (243cm x 121cm)

\usepackage{relsize} % For \smaller
\usepackage{url}     % For \url

\graphicspath{{assets/}} % Root directory of the pictures 

% Colors based on the Omni theme
\definecolor{background}{HTML}{2d2442}
\definecolor{comment}{RGB}{72,60,103}
\definecolor{cyan}{HTML}{86ccd9}
\definecolor{foreground}{RGB}{225,225,230}

\begin{document}
	\begin{poster}{
		grid=false,
		% Background
		background=shadeTB,
		bgColorOne=foreground,
		bgColorTwo=cyan,
		% Poster box
		headershape=rounded,
		headerColorOne=background,
		headerColorTwo=background,
		headerFontColor=foreground,
		headerfont=\Large\bf,
		textborder=faded,
		borderColor=background,
		headerborder=open
	}
	% Eye Cacther
	{}
	% Title
	{\bf
		\textcolor{background}{Testing a new algorithm for isometric embedding of black hole horizons}
	}
	% Authors
	{
		\vspace{1em}
		\textcolor{background}{Iago Braz Mendes, Robert Owen\\
		{\smaller ibrazmen@oberlin.edu, rowen@oberlin.edu}}
	}
	% Logo
	{
		\setlength\fboxsep{0pt}
		\setlength\fboxrule{0.5pt}
		\begin{minipage}{15em}
			\includegraphics[width=11em,height=4em]{oberlin-logo}
			\includegraphics[width=4em,height=4em]{sxs-logo}
		\end{minipage}
	}
		\begin{posterbox}[name=abstract,column=0]{Abstract}
			Isometric Embedding is a classic problem in differential geometry and general relativity that involves constructing a surface in Euclidean space described by a metric tensor. The results from this problem have a long history for visualization, but are also relevant for calculating quantities like black hole mass and energy. Unfortunately, in general scenarios, this problem requires a solver capable of handling a system of strongly nonlinear and nonstandard PDEs, for which there is no generally established algorithm. We have explored a radically new approach to the embedding problem, applying it to a variety of specific test cases and confirming that the results converge as expected and agree with results obtained analytically and by other algorithms. This poster presents the results of a finite-difference-based C++ code that we have written to implement and test this novel algorithm.
		\end{posterbox}

		\begin{posterbox}[name=introduction,column=0,below=abstract,above=bottom]{Introduction}
		\end{posterbox}

		\begin{posterbox}[name=method,column=1,span=2]{Numerical Method}
			\vspace{25em}
		\end{posterbox}

		\begin{posterbox}[name=convergence,column=1,below=method,above=bottom]{Convergence Tests}
		\end{posterbox}

		\begin{posterbox}[name=application,column=2,below=method,above=bottom]{Application to BBHS}
		\end{posterbox}

		\begin{posterbox}[name=future,column=3]{Future Work}
			\vspace{20em}
		\end{posterbox}

		\begin{posterbox}[name=conclusion,column=3,below=future]{Conclusion}
			\vspace{20em}
		\end{posterbox}

		\begin{posterbox}[name=references,column=3,below=conclusion,above=bottom]{References}
		\end{posterbox}
	\end{poster}
\end{document}
